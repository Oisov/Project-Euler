
\begin{ProjectEuler}[Largest prime factor]{3}
The prime factors of $13195$ are $5$, $7$, $13$ and $29$. \medskip

\noindent What is the largest prime factor of the number $600851475143$?
\end{ProjectEuler}

This is another rabbit hole that goes deep, so be warned. However we will stop before you accidentally 
ends up with a masters degree in mathematics. While checking if a number is prime is easy, prime factorization on the other hand is hard. 
Said very naively this is because we do not need to find a single factor to ensure that a number is prime. Instead the most common
technique is a probabilistic approach. However we will come back to that later. 

Instead we will look at a few simple algorithms for factoring primes. At the end I will give some pointers on more complicated
algorithms which is used to factor bigger and harder composite numbers.

Note that the problem only asks us to find the biggest factor. However in term of running time this is just as hard as finding all the factors. 
The most naive approach can be implemented as follows. 
%
\begin{pythoncode}
def largest_primefactor_naive(num):
    i = 2
    while i < num:
        while num % i == 0:
            num //= i
        i += 1
    return num
\end{pythoncode}
%
We can improve the running time in several ways. The first is an improvement of the upper bound
\begin{lemma}
\label{lemma:q<sqrt(n)}
Every composite number $n$ has exactly one prime factor $q$ such that $q > \sqrt{n}$.
\end{lemma}
\begin{proof}
	We will prove this using contradiction. Since $n$ is composite, $n$ has \emph{atleast} two prime factors $p$ and $q$. 
	Assume by contradiction that $p > \sqrt{n}$ and $q > \sqrt{n}$. This implies that $p \cdot q > \sqrt{n} \cdot \sqrt{n} > n$.
	We have now reached a contradiction because: $p$ and $q$ are prime factors of $n \quad \Longleftrightarrow \quad  n = p \cdot q$.
\end{proof}
%
\Cref{lemma:q<sqrt(n)} implies that we only need to test numbers up to $\sqrt{n}$. The last factor $q > \sqrt{n}$ must be a prime factor
otherwise we could have written it as $p = a \cdot b$ and $a, b < \sqrt{n}$. So $a$, $b$ would have been discovered as factors
since we test numbers up to $\sqrt{n}$.

Another improvement is that we do not need to check \emph{every} number up to $\sqrt{n}$. If we first check that the number is divisible by 
$2$ we only need to iterate over the odd numbers.
%
Which of these two methods do you think gives the greatest improvement in speed? 
%
%
\begin{table}[h!tbp]
	\caption{Timings in \si{\ms} for some of the prime factor methods }
	\label{table: PE-3: naive-sqrt-wheel}
	\begin{tabular}{
	c 
	S[table-format=15.0]
	S[table-number-alignment=right, table-figures-decimal=3, table-auto-round]
	S[table-number-alignment=right, table-figures-decimal=3, table-auto-round]
	S[table-number-alignment=right, table-figures-decimal=3, table-auto-round]
	S[table-number-alignment=right, table-figures-decimal=3, table-auto-round]
	S[table-number-alignment=right, table-figures-decimal=3, table-auto-round]
	c }
		\toprule
            &    {Number} & {  Naive } & {sqrt(n)} & {wheel-2} & {wheel-23} & {wheel-235} & \\
        \midrule
            &                600851    &    4.95419670719 & 0.14108598771 & 0.031280533569 & 0.0300534137255 & 0.011960684135 & \\
            &              60085147    &    124.833852142 & 1.24595861474 & 0.213523528579 & 0.237041557598 & 0.116548277881 & \\
            &            6008514751    &    1047.40761616 & 2.80566540182 & 0.799210475336 & 0.651435846951 & 0.416251968787  &\\
          	&          600851475143    &    5.2521231679  & 2.72888239623 & 0.594259930747 & 0.543554725537 & 0.292601595386 &\\
            &      6008514751436008    &    {timeout} 	  & 158.619660114 & 71.0721562764  & 81.5635051611  & 47.0534945511  &  \\
        \bottomrule
    \end{tabular}
\end{table}
%
From \cref{table: PE-3: naive-sqrt-wheel} we can see that the different wheel-factoring algorithms outperforms 
the naive implementation by several orders of magnitude. It is difficult to see exactly how much of an improvement
\verb|wheel-23| is over \verb|wheel-2|, because our numbers are too small. However \verb|wheel-235| offers a significant improvement. 
We can genreralize the \verb|verb| algorithm to $n$-primes using the following code
%
\begin{pythoncode}
	from primesieve import generate_n_primes

	def largest_factor(n, number_of_spokes = 3):
	
	    spokes = generate_n_primes(number_of_spokes)
	    wheel = reduce(lambda x, y: x * y, spokes)
	    composites = []
	    for i in range(1, wheel):
	        if all( i % k != 0 for k in spokes):
	            composites.append(i)
	
	    for x in spokes:
	        if n % x == 0 and n > x:
	            n //= x
	            while n % x == 0:
	                n //= x
	            if n == 1: 
	                return x
	    x = 0
	    limit = int(n**0.5) + 1
	    while x < limit:
	        for k in composites:
	            if n % (x+k) == 0 and x+k > 1:
	                n //= (x+k)
	                while n % (x+k) == 0:
	                    n //= (x+k)
	                if n == 1: return x+k
	                limit = int(n**0.5) + 1
	        x += wheel
	    return n
\end{pythoncode}
%
We can improve the \verb|wheel| factorization further by only iterating over the primes. This is the best we can do for a wheel factorization, however
it is only best given that we have a fast way to generate primes. Keeping a large list of primes would slow this method down significantly.
Luckily Python has a few libraries which can help us solve this problem.
%
\begin{pythoncode}
	from primesieve import Iterator
	
	def prime_gen(num):
	    if isprime(num): return num
	    it = Iterator()
	    prime = it.next_prime()
	    limit = num**0.5
	    while prime < limit:
	        if num % prime == 0: 
	            num //= prime
	            while num % prime == 0: num //= prime
	            if num == 1: return prime
	            if isprime(num): return num
	            limit = num**0.5
	        prime = it.next_prime()
	    return num
\end{pythoncode}
%
As we shall see for small numbers this gives a significant speedboost. Until now all of our methods have consisted of sieving out primes
starting with the smallest one. As you probably have seen this is slow for all but the tiniest of numbers. To get another speed improvement
we need to change our approach. One of the simplest alternatives to the wheel-factorization is the Pollard's rho algorithm
\footnote{See \url{https://en.wikipedia.org/wiki/Pollard\%27s_rho_algorithm}}.
%
\paragraph*{Pollard's rho algorithm}
%
\begin{pythoncode}
def pollard_rho(N):
        if N%2==0:
                return 2
        x = randint(1, N-1)
        y = x
        c = randint(1, N-1)
        g = 1
        while g==1:             
                x = ((x*x)%N+c)%N
                y = ((y*y)%N+c)%N
                y = ((y*y)%N+c)%N
                g = gcd(abs(x-y),N)
        return g
\end{pythoncode}
%
This returns just a single divisor. To get all the divisors we can use a simple generator
%
\begin{pythoncode}
	def generator(N):
	    while not isprime(N) and N > 1:
	        factor = pollard_rho(N)
	        for fac in pollard_rho_generator(factor):
	            yield fac
	        N //= factor
	    yield N
\end{pythoncode}
%
We will use this generator for a few other functions as well. For now it is enough
to say that it recursively generates the prime factors. A quick explanation of how this algorithm 
works can be found in the footnotes\footnote{The Birthday Paradox: A Quick Tutorial on Pollard's Rho Algorithm
\url{https://www.cs.colorado.edu/~srirams/courses/csci2824-spr14/pollardsRho.html}}  

Brent's factorization method is an improvemen to Pollard's rho algorithm,
published by R. Brent in 1980 [9]. In Pollard's rho algorithm, one tries to find a 
non trivial factor $s$ of $N$ by finding indices $i$, $j$ with $i < j$ such that $x_i \equiv \pmod{s}$ and
$x_i \not\equiv x_j \pmod{N}$. The $x_n$ sequence is defined by the recurrence relation:
\begin{align*}
	x_0 & \equiv 2 \pmod{N} \\
	x_{n+1} & \equiv x_{n}^2 + 2 \pmod{N}
\end{align*}
Pollard suggested that $x_n$ be compared to n $x_{2n}$ for $n = 1,\,2,\,3,\,\ldots$ Brent's
improvement to Pollard's method is to compare $n_x$ to $x_m$ , where $m$ is the largest integral
power of $2$ less than $n$.
%
\paragraph*{Brent's factorization method}
%
\begin{pythoncode}
	def brent(num):
	    if num % 2 == 0:
	        return 2
	    y, c, m = randint(1, num-1), randint(1, num-1), randint(1, num-1)
	    s, r, q = 1, 1, 1
	    while s == 1:
	        x = y
	        for i in range(r):
	            y = ((y*y) % num+c) % num
	        k = 0
	        while (k < r and s == 1):
	            ys = y
	            for i in range(min(m, r-k)):
	                y = ((y*y) % num+c) % num
	                q = q*(abs(x-y)) % num
	            s = gcd(q, num)
	            k += m
	        r = r*2
	    if s == num:
	        while True:
	            ys = ((ys*ys) % num+c) % num
	            s = gcd(abs(x-ys), num)
	            if s > 1:
	                break
	    return s
\end{pythoncode}
%
We can make a slight improvement to this algorithm. Most fast prime factorization algorithms start with sieving 
out the small prime factors. 
%
\begin{pythoncode}
	from primesieve import generate_n_primes
	
	NUM_OF_PRIMES = 10**3
	PRIMES = generate_n_primes(NUM_OF_PRIMES)
	
	def small_factor_generator(num):
	    if isprime(num):
	        yield num
	    for prime in PRIMES:
	        if num % prime == 0:
	            num //= prime
	            yield prime  # Returns the next factor
	            while num % prime == 0:
	                num //= prime
	                yield prime
	            if prime > num**0.5:
	                break
	    yield num
\end{pythoncode}
%
As usual Python has specialized algorithms for factoring integers, one of these is the \verb|primefac| 
package\footnote{\url{https://pypi.python.org/pypi/primefac}}. The description of the package reads as follows:

\begin{displayquote}
	This is a module and command-line utility for factoring integers. As a module, we provide a primality test, several functions 
	for extracting a non-trivial factor of an integer, and a generator that yields all of a number’s prime factors (with multiplicity). 
	As a command-line utility, this project aims to replace GNU’s factor command with a more versatile utility — in particular, 
	this utility can operate on arbitrarily large numbers, uses multiple cores in parallel, uses better algorithms, handles input 
	in reverse Polish notation, and can be tweaked via command-line flags. Specifically
%
	\begin{itemize}
	    \item One thread runs Brent’s variation on Pollard’s rho algorithm. This is good for extracting smallish factors quickly.
	    \item One thread runs the two-stage version of Pollard’s p-1 method. This is good at finding factors p for which p-1 is a 
	    	  product of small primes.
	    \item One thread runs Williams’ p+1 method. This is good at finding factors p for which p+1 is a product of small primes.
	    \item One thread runs the elliptic curve method. This is a bit slower than Pollard’s rho algorithm when the factors 
	    	  extracted are small, but it has significantly better performance on difficult factors.
	    \item One thread runs the multiple-polynomial quadratic sieve. This is the best algorithm for factoring "hard" numbers 
	    	  short of the horrifically complex general number field sieve. However, it’s (relatively speaking) more than a little 
	    	  slow when the numbers are small, and the time it takes depends only on the size of the number being factored rather 
	    	  than the size of the factors being extracted as with Pollard’s rho algorithm and the elliptic curve method, so we use 
	    	  the preceding algorithms to handle those.
	\end{itemize}
%
\end{displayquote}

There has been much studies done on using elliptic curves for factorizing integers. These methods in general are quite complex to not only
implement but also understand. I implemented a naive version of the Lenstra elliptic curve factorization \cite{Lenstra}, however
this did not give any speed increases on the integers tested. The reason is that the elliptic curve I used was choosen at random. 
The \verb|primefac| package probably has a much better implementation than me.

Some of the heavier methods are much slower for small factor. The following list gives a rough estimate for which algorithm should be used
at what number range. 
%
\begin{itemize}
	\item Small Numbers : Use simple sieve algorithms to create list of primes and do plain factorization. Works blazingly fast for small numbers.
	\item Big Numbers : Use Pollard's rho algorithm, Shanks' square forms factorization (Thanks to Dana Jacobsen for the pointer)
	\item Less Than $10^{25}$ : Use Lenstra elliptic curve factorization \cite{Lenstra}
	\item Less Than $10^{100}$ : Use Use Quadratic sieve
	\item More Than $10^{100}$ : See \cite{Pomerance96atale} for a layman introduction to the General number field sieve [GNFS]. For a more in depth study
	of GNFS see \cite{Briggs98anintroduction}.
\end{itemize}
%
A speed comparison for the more advanced factoring methods can be found in \cref{table: PE-3: primegen-pollard-brent-primefac}. 
Brent* denotes, uses a prime sieve to remove small primefactors before invoking Brent's improved Pollard Rho algorithm. 
%
\begin{table}[h!tbp]
	\caption{Timings in \si{\ms} for some of the prime factor methods }
	\label{table: PE-3: primegen-pollard-brent-primefac}
	\begin{tabular}{
	c 
	S[table-format=23.0]
	S[table-number-alignment=center, table-figures-decimal=3, table-auto-round]
	S[table-number-alignment=center, table-figures-decimal=3, table-auto-round]
	S[table-number-alignment=center, table-figures-decimal=3, table-auto-round]
	S[table-number-alignment=center, table-figures-decimal=3, table-auto-round]
	c }
		\toprule
            &                 {Number}   &        {Prime gen}  &  {Pollard Rho} &        {Brent} &        {Brent*} &    \\
        \midrule
          	&             600851475143    &    0.267652838367  & 0.303145397497 & 0.241768215367 & 0.0817887596012 &  \\
            &         6008514751436008    &    0.245708782489  & 0.243921040691 & 0.215603781952 & 0.232111869427  & \\
            & 600851475143600851475143    &    1.63928404798   & 1.95475265024  & 1.04038712453  & 0.366906164083  & \\
            &            1389133318189    &    34.9505326148   & 3.58912515518  & 1.42747487672  & 1.72358896308   & \\
            &          138912436076543    &    284.297268737   & 11.0395493611  & 3.7835773567   & 3.89397909398   & \\
            &        13891248322099591    &    \text{timeout}  & 103.696743328  & 20.749950072   & 14.7731517139   & \\ 
        \bottomrule
    \end{tabular}
\end{table}
%
\bibliographystyle{amsalpha}
\bibliography{../REFF}