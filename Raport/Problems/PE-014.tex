

\begin{ProjectEuler}[Lattice paths]{15}
  Starting in the top left corner of a 2×2 grid, and only being able to move to
  the right and down, there are exactly 6 routes to the bottom right corner.
  %
  \begin{center}
    \includegraphics[scale=0.5]{../Images/PE_015.png}
  \end{center}
  %
  How many such routes are there through a $20\times20$ grid?
\end{ProjectEuler}

\subsection{Recursive solution}

\lstinputlisting[language=Julia, 
                 caption=Recursive route-counting function,
                 ]{../Problems/PE_015/Julia/PE_015_recursive.jl}

To speed up this code we could have defined
\lstinline|path_sum[(m,n)]=path_sum[(n,m)]| and as we know that there are only
one way to reach each of the points on the first column and first row. However,
while this halves the number of function calls this is not where the bottleneck
lies. Most of the time is spent looking up dictionary values. 

\newpage

\subsection{Iterative solution}

While the memoized recursion solves the problem, it is not only slow, but
requires more memory, and some languages have trouble with deeply-nested
recursive calls.

If we could translate our recursion into and \emph{iterative} solution using
bottom-up programming technique called \emph{dynamic programming} we would solve
most of the concerns raised above. 
\lstinputlisting[language=Julia, 
                 caption=Recursive route-counting function,
                 ]{../Problems/PE_015/Julia/PE_015_iterative.jl}

\subsection{Combinatorial solution}
                 
\lstinputlisting[language=Julia, 
                 caption=Recursive route-counting function,
                 ]{../Problems/PE_015/Julia/PE_015.jl}

%%% Local Variables:
%%% TeX-master: "../ProjectEuler"
%%% End: